\newpage
\section{Auswertung}
Die im Folgenden genannten Intensitäten ergeben sich aus dem Flächeninhalt des Peaks bei einer Energie von $661,7\,\text{eV}$
im Absorptionsspektrum, da es sich dabei um die Energie der verwendeten Cs-Lampe handelt \cite{gilmore_practical_2008}.

Die gemessenen Intensitäten für die vier ausgemessenen Würfel sind in Tabelle \ref{tab:Werte} aufgetragen.
Dabei beschreiben die Intensitäten für den ersten Würfel, also nur die Aluminium-Hülle, die Grundintensität $I_0$.
\begin{center}
    \captionof{table}{Messwerte der Absorptionsmessungen für eine Messzeit von jeweils $\Delta t=300\,\text{s}$.
                        $i$ beschriebt die in Abbildung \ref{fig:Richtungen} gezeigten Bestrahlungsrichtungen.}
    \label{tab:Werte}
    \begin{tabular}{c c c c c}
        \toprule
        $i$ & $I_0$ (Al-Hülle) & $I_i$ (Würfel 2) & $I_i$ (Würfel 3) & $I_i$ (Würfel 4) \\
        \midrule
        1  & (56088$\pm$279) & (47413$\pm$251) & (1800$\pm$21) & (46723$\pm$254) \\
        2  & (55842$\pm$282) & (47567$\pm$265) & (1033$\pm$51) & (184$\pm$31) \\
        3  & (56650$\pm$292) & (47079$\pm$268) & (1346$\pm$60) & (44259$\pm$264) \\
        4  & (55866$\pm$290) & (47204$\pm$266) & (2670$\pm$70) & (12147$\pm$130) \\
        5  & (55383$\pm$284) & (44569$\pm$252) & (612$\pm$13)  & (9232$\pm$116) \\
        6  & (54930$\pm$275) & (46910$\pm$253) & (2257$\pm$60) & (11316$\pm$129) \\
        7  &  &  &  & (11028$\pm$158) \\
        8  &  &  &  & (15401$\pm$148) \\
        9  &  &  &  & (15514$\pm$150) \\
        10 &  &  &  & (12527$\pm$135) \\
        11 &  &  &  & (8965$\pm$120) \\
        12 &  &  &  & (10907$\pm$127) \\
        \bottomrule
    \end{tabular}
\end{center}

Um die Absoptionskoeffizienten der verschiedenen Würfelmaterialien bestimmen zu können, muss für die drei Messreihen der Intensität $I_i$
das Gleichungssystem \ref{unsere_matrix} gelöst werden.
Dabei handelt es sich in der Regel um ein nicht-quadratisches lineares Gleichungssystem.
Dieses wird mit der Methode kleinster Quadrate auf eine quadratische Form angenähert und dann nach den Absorptionskoeffizienten $\mu_j$ gelöst.

Das Ergebnis dieses Verfahrens für die beiden homogen gefüllten Würfel ist in Tabelle \ref{tab:Erg} zu sehen.
Da es sich dabei eben um homogene Würfel handelt und diese lediglich mit den Projektionen $i=1,...,6$ vermessen wurden (siehe Tabelle \ref{tab:Werte}),
wird die Geometriematrix aus Gleichung \ref{unsere_matrix} in diesem Fall wie folgt vereinfacht.
\begin{equation}
    \begin{pmatrix}
        1 & 1 & 1 & 0 & 0 & 0 & 0 & 0 & 0 \\
        0 & 0 & 0 & 1 & 1 & 1 & 0 & 0 & 0 \\
        0 & 0 & 0 & 0 & 0 & 0 & 1 & 1 & 1 \\
        0 & \sqrt{2} & 0 & 0 & 0 & \sqrt{2} & 0 & 0 & 0 \\
        \sqrt{2} & 0 & 0 & 0 & \sqrt{2} & 0 & 0 & 0 & \sqrt{2} \\
        0 & 0 & 0 & \sqrt{2} & 0 & 0 & 0 & \sqrt{2} & 0 \\
        0 & 0 & 1 & 0 & 0 & 1 & 0 & 0 & 1 \\
        0 & 1 & 0 & 0 & 1 & 0 & 0 & 1 & 0 \\
        1 & 0 & 0 & 1 & 0 & 0 & 1 & 0 & 0 \\
        0 & 0 & 0 & 0 & 0 & \sqrt{2} & 0 & \sqrt{2} & 0 \\
        0 & 0 & \sqrt{2} & 0 & \sqrt{2} & 0 & \sqrt{2} & 0 & 0 \\
        0 & \sqrt{2} & 0 & \sqrt{2} & 0 & 0 & 0 & 0 & 0 
      \end{pmatrix}
      \qquad \Longrightarrow \qquad
      \begin{pmatrix}
        3 \\
        3 \\
        3 \\
        2\sqrt{2} \\
        3\sqrt{2} \\
        2\sqrt{2} \\
      \end{pmatrix}
\end{equation}
% \newpage
\begin{center}
    \captionof{table}{Berechnete Absorptionskoeffizienten für die neun Teilwürfel der zwei homogonen Würfel.}
    \label{tab:Erg}
    \begin{tabular}{c c c c c}
        \toprule
        $\mu$ / $\text{cm}^{-1}$ (Würfel 2) & Abw. / \% (Delrin) & $\mu$ / $\text{cm}^{-1}$ (Würfel 3) & Abw. / \% (Blei) \\
        \midrule
        (0,055$\pm$0.003) & 52,16 & (1.152$\pm$0.010) & 7,85 \\
        \bottomrule
    \end{tabular}
\end{center}
Von den uns gegebenen möglichen Materialien, ähnelt Würfel 2 vom Absorptionskoeffizienten her Delrin ($\mu_{Delrin}=0.116\,\text{cm}^{-1}$)
und Würfel 3 dem von Blei ($\mu_{Blei}=1.2497\,\text{cm}^{-1}$). Die Abweichungen der Absorptionskoeffizienten sind ebenfalls in der Tabelle ablesbar.

Da im Falle des letzten Würfels in in allen 12 Orientierungen gemessen wurde ($i=1,...,12$),
die Grundintensität $I_0$ jedoch lediglich in 6 Richtungen ($i=1,...,6$) (siehe Tabelle \ref{tab:Werte}) gemessen wurde,
wurden die Messwerte für die fehlenden 6 Orientierungen ($i=6,...,12$) beim leeren Würfel von den ersten 6 übernommen.
Diese Einzelmessungen sind daher nicht mehr unabhängig voneinander.
Unter Beachtung der Korrelation dieser Einzelmessungen, wurde die Lösung für die Absorptionskoeffizienten
mittels der Methode kleinster Quadrate gefunden und ist in Tabelle \ref{tab:Erg2} aufgetragen.
% \begin{center}
%     \captionof{table}{Berechnete Absorptionskoeffizienten für den inhomogen gefüllten Würfel.}
%     \label{tab:Erg2}
%     \begin{tabular}{c c c c}
%         \toprule
%         $j$ & $\mu_j$ / $\text{cm}^{-1}$ (Würfel 4) & Material & Abweichung / \% \\
%         \midrule
%         1 & (-0.1711$\pm$0.0088) & / & / \\
%         2 & (-0.2806$\pm$0.0172) & / & / \\
%         3 & (0.0360$\pm$0.0062)  & Delrin & 68,97$\,$\% \\
%         4 & (1.6296$\pm$0.0374)  & Blei & 30,40$\,$\% \\
%         5 & (1.6441$\pm$0.0396)  & Blei & 31,56$\,$\% \\
%         6 & (1.5675$\pm$0.0370)  & Blei & 25,43$\,$\% \\
%         7 & (-0.1167$\pm$0.0082) & / & / \\
%         8 & (-0.3045$\pm$0.0172) & / & / \\
%         9 & (0.0697$\pm$0.0062)  & Delrin & 39,91$\,$\% \\
%         \bottomrule
%     \end{tabular}
% \end{center}
\newpage
\begin{center}
    \captionof{table}{Berechnete Absorptionskoeffizienten für den inhomogen gefüllten Würfel.}
    \label{tab:Erg2}
    \begin{tabular}{c c c c}
        \toprule
        $j$ & $\mu_j$ / $\text{cm}^{-1}$ (Würfel 4) & Material & Abweichung / \% \\
        \midrule
        1 & (-0,17$\pm$0,08) & / & / \\
        2 & (-0,28$\pm$0,08) & / & / \\
        3 & (0,04$\pm$0,08)  & Delrin & 68,97 \\
        4 & (1,63$\pm$0,10)  & Blei & 30,40 \\
        5 & (1,64$\pm$0,11)  & Blei & 31,56 \\
        6 & (1,57$\pm$0,10)  & Blei & 25,43 \\
        7 & (-0,12$\pm$0,08) & / & / \\
        8 & (-0,30$\pm$0,08) & / & / \\
        9 & (0,07$\pm$0,08)  & Delrin & 39,91 \\
        \bottomrule
    \end{tabular}
\end{center}
Den negativen Lösungen für die Absoptionskoeffizienten werden keine Materialien zugeordnet.
Auf mögliche Gründe dafür wird in der Diskussion eingegangen.
Die Materialien und die Abweichungen für die restlichen Absoptionskoeffizienten sind in Tabelle \ref{tab:Erg2} eingetragen.