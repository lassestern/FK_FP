\newpage
\section{Diskussion}    
    Im Rahmen des Versuches konnten alle CdSe-Proben und die Kohlenstoff Nanopartikelprobe zu Photolumineszenz angeregt werden. Während der Untersuchung der CdSe-Proben mit dem \SI{405}{\nano\metre} konnten
    die Durchmesser aller Nanokristalle auf Werte zwischen \SI{2.50}{\nano\metre} und \SI{1.65}{\nano\metre} besti2mmt werden. Diese Werte befinden sich alle innerhalb eines plausiblen Bereiches für 
    Nanokristalle, erscheinen jedoch im Vergleich mit Abbildung \ref{fig:Farbe} circa um einen Faktor 2 zu klein. Die Abweichung um einen genauen Faktor lässt vermuten, dass in der Messung oder Auswertung ein 
    systematischer Fehler vorleigt, der jedoch nach längerer Korrektur nicht gefunden werden konnte. 
    
    Neben der Größe der Nanokristalle wurde über die Messung der polarisationsabhängigen Intensität der Photolumineszenz auch gezeigt, dass die photolumineszenzte Strahlung unpolarisiert ist. Dies ist zu 
    erwarten, da die Nanokristalle zufällig in der Flüssigkeit orientiert sind und somit im Mittel isotrop und nicht polarisiertes Licht aussenden.

    Die Abhängigkeit der Photolumineszenzintensität von der Laserleistung zeigt einen linearen Zusammenhang dieser mit der Anregungsleistung, aus dem noch kein Sättigungsverhalten zu erkennen ist, wie es 
    eigentlich für die Erzeugung von immer mehr Exzitonen zu erwarten wäre. Aufällig ist in dieser Messung ein Absacken der Intensität und das anschließend wieder lineare Wachstum dieser. Der Sprung ist 
    theoretisch nicht zu erklären und wird vermutlich auf die Photodiode zur Messung der Anregungsleistung zurückzuführen sein. Womöglich wurde dieser an der Sprungstelle neu genullt oder von dort an leicht
    verändert in den Laserstrahl gehalten. 
    
    Die Messung der Photolumineszenzspektren der verschiedenen Proben mit unterschiedlichen Anregungsenergien zeigt die Abhängigkeit der Energie des Exzitons vom Durchmesser des Nanokristalls, da für kleinere 
    Kristalle keine Anregung mit niedrigeren Anregungsenergien mehr möglich ist. 

    Zusammenfassend ist es möglich die prinzipiellen Eigenschaften und Verhaltensweisen von Nanokristallen, wie deren Größe und deren Photolumineszenzverhalten, mit dem vorliegenden Aufbau zu unteruschen.
    In diesem speziellen Experiment liefert nur die Messung der Größe einen Abweichungsfaktor von 2, der nicht geklärt werden konnte.  