\section{Diskussion}    
    \subsection{Topografie einer Mikrostruktur}
        Wie schon in Abbildung \ref{fig:Mikrostruktur} gezeigt ist die Mikrostruktur aufgeteilt in zwei verschiedene Bereiche. Die Kreis- und Streifenstrukturen haben laut Datenblatt eine $5\,\mu\text{m}$-Periodizität und die Quadratstruktur eine $10\,\mu\text{m}$-Periodizität. In Tabelle \ref{tab:MikroLit} sind die die bereits in Kapitel \ref{sec:Mikro} berechneten Strukturperiodizitäten und die Abweichungen von den im Datenblatt gegebenen Größen aufgetragen.
        \begin{center}
            \captionof{table}{Periodizitäten der drei Bereiche der zu untersuchenden Probe einschließlich Abweichungen von den erwarteten Werten.}
            \label{tab:MikroLit}
            \begin{tabular}{c c c c}
                \toprule
                Mikrostruktur & Periodizität (exp.) / $\mu\text{m}$ & Periodizität (lit.) / $\mu\text{m}$ & Abweichung\\
                \midrule
                Kreis (X)   & $4.737$  & $5$ & $5.26\%$ \\
                Kreis (Y)   & $5.080$  & $5$ & $1.60\%$ \\
                Quadrat (X) & $9.437$  & $10$& $5.63\%$ \\
                Quadrat (Y) & $10.155$ & $10$& $1.55\%$ \\
                Streifen (X)& $4.740$  & $5$ & $5.20\%$ \\
                \bottomrule
            \end{tabular}
        \end{center}
        Die Abweichungen sind klein genug um sagen zu können, dass die Mikrostruktur erfolgreich vermessen wurden konnte. Außerdem ähneln sich die x-Abweichungen bzw. die y-Abweichungen untereinander. Diese können möglicherweise auf einen systematischen Fehler zurückzuführen werden kann, wie zum Beispiel einer leichten Verkippung der Probe um die x-Achse und einer etwas stärkeren Verkippung um die y-Achse.
        \FloatBarrier
    \subsection{Topografie einer CD, DVD und Blu-Ray}
        Die gemessenen Größen der CD sind in Tabelle \ref{tab:CDlit} mit Literaturwerten und der resultierenden Abweichung dargestellt. Der Wert für die erwartete Pittiefe resultiert daraus, dass diese einem Viertel der Wellenlänge des auftreffenden Lichts entsprechen sollte. Der zum Lesen einer CD verwendete Laser hat normalerweise eine Wellenlänge von $\lambda=\SI{780}{\nano\metre}$. Das entspricht, aufgrund des Brechungsindex der Polycarbonatbeschichtung von $n_{PC}=1.58$, einer effektiven Wellenlänge von $\lambda=\SI{493.67}{\nano\metre}$ und damit einer erwarteten Pittiefe von $\SI{123.42}{\nano\metre}$.
        \newpage
        \begin{center}
            \captionof{table}{Vergleich der gemessenen Größen der CD mit den erwarteten Größen aus der Literatur. \cite{CD}}
            \label{tab:CDlit}
            \begin{tabular}{c c c c}
                \toprule
                CD & gemessen & erwartet & Abweichung \\
                \midrule
                Spurbreite / $\mu\text{m}$    & ($0.424\pm 0.024$) & $0.5$   & $15.2\,\%$ \\
                Spurabstand / $\mu\text{m}$   & ($1.123\pm 0.032$) & $1.1$   & $2.09\,\%$ \\
                min. Pitlänge / $\mu\text{m}$ & ($0.684\pm 0.013$) & $0.833$ & $17.89\,\%$ \\
                max. Pitlänge / $\mu\text{m}$ & ($1.715\pm 0.035$) & $3.054$ & $43.84˜,\%$ \\
                Pittiefe / $\text{nm}$         & ($22.141\pm 0.448$)& $123.42$& $82.06\,\%$ \\
                \bottomrule
            \end{tabular}
        \end{center}
        Die größere Abweichung bei der maximalen Pitlänge lässt sich dadurch rechtfertigen, dass nicht davon ausgegangen werden kann, dass der längste mögliche Pit mit dem $10\times 10\,\mu\text{m}$ Fenster erfasst wurde. Die starke Abweichung bei der Pittiefe kann z.B. auf eine abgenutzte Probenoberfläche zurückzuführen sein.
        Die ermittelte maximale Speicherkapazität hat ebenfalls eine starke Abweichung von einem Faktor $\approx 4$ und kann möglicherweise aus einer falsch verwendeten Formel stammen.

        Die Abweichungen der Messungen zu den Literaturwerten für die DVD und BluRay Proben sind in Tabelle \ref{tab:DVDlit} und \ref{tab:BRlit} wie für die CD erklärt dargestellt.
        \begin{center}
            \captionof{table}{Vergleich der gemessenen Größen der DVD mit den erwarteten Größen aus der Literatur. \cite{DVD1} \cite{DVD2}}
            \label{tab:DVDlit}
            \begin{tabular}{c c c c}
                \toprule
                DVD & gemessen & erwartet & Abweichung \\
                \midrule
                Spurperiodizität / $\mu\text{m}$   & ($0.841\pm 0.053$) & $0.74$ & $13.65\,\%$ \\
                min. Pitlänge / $\mu\text{m}$ & ($0.283\pm 0.022$) & $0.4$ & $29.25\,\%$ \\
                max. Pitlänge / $\mu\text{m}$ & ($0.990\pm 0.001$) & $1.87$ & $47.06\,\%$ \\
                Pittiefe / $\text{nm}$        & ($11.753\pm 0.265$)& $102.8$& $82.06\,\%$ \\
                \bottomrule
            \end{tabular}
        \end{center}
        \begin{center}
            \captionof{table}{Vergleich der gemessenen Größen der BluRay mit den erwarteten Größen aus der Literatur. \cite{BluRay}}
            \label{tab:BRlit}
            \begin{tabular}{c c c c}
                \toprule
                BluRay & gemessen & erwartet & Abweichung \\
                \midrule
                Spurperiodizität / $\mu\text{m}$   & ($0.308\pm 0.008$) & $0.32$   & $3.75\,\%$ \\
                min. Pitlänge / $\mu\text{m}$ & ($0.183\pm 0.010$) & $0.149$ & $22.82\,\%$ \\
                max. Pitlänge / $\mu\text{m}$ & ($0.576\pm 0.003$) & $0.695$ & $17.21\,\%$ \\
                Pittiefe / $\text{nm}$        & ($8.210\pm 0.211$) & $123.42$& $82.06\,\%$ \\
                \bottomrule
            \end{tabular}
        \end{center}
        \FloatBarrier
    \subsection{Adhäsionskraft und Elastizitätsmodul}
        Die berechneten Adhäsionskräfte liegen alle bei ca. $\SI{100}{\nano\newton}$, wobei bei Edelstahl die größte Kraft aufgewendet werden muss, um die Spitze von den Probe zu lösen.

        Das bestimmte Elastizitätsmodul $E = \SI{20.86}{\mega\pascal}$ von Teflon weist eine starke Abweichung vom Literaturwert des "Young-Moduls"  $E_Y = \SI{575}{\mega\pascal}$ auf. Er ähnelt viel mehr der Kompressionsstärke von Teflon, welche bei $\SI{23.5}{\mega\pascal}$ \cite{Teflon} liegt. Den potentiellen Grund dafür können wir uns jedoch nicht erklären.