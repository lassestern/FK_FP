\newpage
\section{Diskussion}    
\label{sec:Dis}
    Der Literaturwert für die Pulsdauer des unmodulierten Lasers leigt bei $\Delta \tau^{\text{Laser,Lit}}=\SI{100}{\femto\second}$. Das entspricht einer Abweichung von $19.59\,\%$ von der vermessenen Pulsdauer $\Delta \tau^{\text{Laser}} = (80.41\pm 0.92)\,\text{fs}$. Diese kann möglichweise darauf zurückgeführt werden, dass die "Peak-Schultern" in der Autokorrelationsspur nicht für die Bestimmung der Pulsbreite betrachtet wurden und aus diesem Grund konnte die Pulsdauer unterschätzt wurde.
    Eine Gauß-Summe als Fit-Funktion lieferte allerdings eine noch höhere Abweichung vom Literaturwert, weshalb ein einfacher Gauß-Fit verwendet wurde.

    Für den Lasers nach Durchlaufen der Bandpassfilter ergaben sich deutlich verbreiterte Pulse. Das stimmt mit Formel \ref{eqn:Heisenberg} überein, dass die minimale Pulsbreite durch die maximale Bandbreite begrenzt ist. Als Faktor für die Pulsverbreiterung ergab sich dabei für den $\SI{30}{\nano\metre}$- bzw. $\SI{12}{\nano\metre}$-Filter $2.1$ bzw. $3.6$. Diese stimmen in etwa mit den Verhältnissen Bandbreiten $\Delta\lambda$ überein, welche aus den Spektren in Abbildungen \ref{fig:normal}, \ref{fig:30} und \ref{fig:12} ermittelt wurden.

    Im Falle der Propagation des Lasers durch Glassblöcke steigender Dicke ergab sich eine steigende Pulsbreite, aufgrund des längeren, im dispersiven Medium zurückgelegten, Weges. Da SF11 Glass einen relativ geringen Brechungsindex ($n_{\text{SF11}}\approx 1.8$) besitzt ist der Effekt nicht so groß und es entsteht lediglich eine Pulsverbreiterung von $5.93 - 16.65 \,\text{fs}$. Anders sieht es bei dem Silizium-Block aus, wo es aufgrund des hohen Brechungsindizes ($n_{\text{Si}}\approx 3.7$) zu einem großen Geschwindigkeitsunterschied zwischen den unterschiedlichen Wellenlängen kommt. Daher tritt schon bei einer Propagationsstrecke von $\SI{12}{\milli\metre}$ eine starke Pulsverbreiterung um den Faktor $\sim 7$ auf.

    Abschließend kann gesagt werden, dass die betimmten Pulsbreiten stets den Erwartungen entsprochen haben.