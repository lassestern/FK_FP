\section{Diskussion}    
    Die besonders großen Abweichungen beim ersten Würfel, können darauf zurückzuführen sein,
    dass es sich bei dem Material um keines der uns gegeben Materialien handelt,
    sondern um eines mit einem noch geringeren Absorptionskoeffizienten, wie z.B. Holz.
    Das würde die Vielzahl an Lösungen erklären mit einem $\mu<0.1\,\text{cm}^{-1}$.

    Bei der Messung beim zweiten Würfel konnte mit einer Messzeit von $300\,\text{s}$ nicht auf eine Intensität von über 10000 Counts gestoßen werden.
    Deswegen weißt diese Messung eine deutlich schlechtere Statistik auf als die Messung davor.
    Dennoch liegen die Messwerte erstaunlich nahe beieinander und die Abweichung zum Absorptionskoeffizienten von Blei ist nicht allzu hoch.

    Beim zusammengesetzten Würfel ist erkenntlich, dass dieser aus zwei verschiedenen Materialien zusammengebaut ist.
    Die Teilwürfel 4,5,6 und 3,9 bestehen jeweils aus dem selben Material und
    diese passen in etwa mit den Materialien der homogenen Würfel zusammen.
    Die teilweise negativen Absorptionskoeffizienten lassen auf Leerstellen im letzten Würfel zurückführen.
    % Ein weiterer Grund könnte der Messwert der dritten Orientierung sein, welcher sehr stark von den anderen abweicht und ein
    % Indiz für fälschliches Ablesen der Messwerte sein kann.

    Im Allgemeinen konnte der Strahlengang lediglich per Augenmaß justiert werden und die Position des Würfels in dem
    Strahlengangkonnte dementsprechend nicht exakt eingestellt werden, was zu systematischen Abweichungen führen konnte.
