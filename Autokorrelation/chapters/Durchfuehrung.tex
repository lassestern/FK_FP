\newpage
\section{Versuchsaufbau und -durchführung}
  Der hier verwendete Versuchsaufbau ähnelt dem in \ref{fig:autokorrelator} gezeigten typischen Autokorrelations-Set-Up. In unserem Fall wird die Linse zur Fokussierung der zwei eingehenden Strahlen auf den BBO-Kristall durch einen sphärischen Spiegel ersetzt. Außerdem wird, wie schon in Kapitel \ref{sec:AK47} beschrieben ein Lock-In-Verstärker sowie ein Chopper verwendet, um Hintergrundrauschen rauszufiltern.

  Zum Vermessen der Autokorrelationsspuren wird zuerst durch grobes Abtasten der ungefähre Mittelpunkt und die Breite des Autokorrelationssignals bestimmt, um den Messbereich der Delay-Line festzulegen. Daraufhin wird die Delay-Line in $\SI{1}{\micro\metre}$-Schritten abgefahren und das Autokorrelationssignal durch die Photodiode vermessen.

  Als Erstes wird der Laser vermessen ohne in irgendeiner Weise moduliert zu werden. Daraufhin werden abwechselnd zwei Bandpassfilter in den Strahlengang zwischen die Laser-Erzeugung und den Beam-Splitter eingesetzt. Diese beschränken die Bandbreite auf einen Bereich von $\pm \SI{30}{\nano\metre}$ bzw. $\pm \SI{12}{\nano\metre}$ um die Wellenlänge von $\SI{1550}{\nano\metre}$, welche der zentralen Wellenlänge des Lasersystems entspricht. Für beide Filter wird jeweils die Autokorrelationsspur aufgenommen.
  Für jede dieser drei Messungen wird ebenfalls ein Spektrum aufgenommen vor dem Eintreffen am BBO-Kristall aufgenommen.

  Als nächstes werden die Filter aus dem Strahlengang entfernt und ein $\SI{12}{\milli\metre}$ dicker Silizium-Block eingesetzt. Nach Abfahren der Delay-Line und einem erneuten Vermessen des Autokorrelationssignals wird der Silizium Block durch SF11-Glassblöcke ersetzt. Es standen vier Blöcke mit den Dicken ($13.51\,\text{mm}$), ($17.64\,\text{mm}$), ($23.85\,\text{mm}$) und ($29.83\,\text{mm}$) zu Verfügung. Diese wurde jeweils im Strahlengang vermessen und zusätzlich wurden die zwei letzten Blöcke kombiniert und es wurde die Autokorrelationsspur für eine Propagationsstrecke $23.85\,\text{mm} + 29.83\,\text{mm} = 53.68\,\text{mm}$ aufgenommen.