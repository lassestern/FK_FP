\section{Ziel}
  Heutzutage ist es mit modernen Lasersystemen möglich ultrakurze Pulse im Bereich von wenigen Femtosekunden zu erzeugen und mit Hilfe dieser Prozesse mit Zeitskalen in der gleichen Größenordnung zu 
  untersuchen. Problematisch stellt sich dabei das Messen dieser ultrakurzen Pulsedar, weil die Messelektronik zu langsam arbeitet und nur Pulse bis zum niedrigen Nanosekundenbereich vermessen kann. In
  der Methode der Autokorrelation wird diese Limitierung umgangen, indem der fs-Puls mit sich selbst abgetastet wird und die Intensität bei jedem Abtastschritt über längere Zeit mit der vorhandenen 
  Elektronik gemessen werden kann. Solch ein Autokorrelator soll in diesem Versuch genutzt werden, um fs-Pulse sowie deren Verhalten beim Transmittieren durch Medien, wie Glas oder Silizium zu untersuchen.
    
\section{Theoretische Grundlagen}

  \subsection{Erzeugung ultrakurzer Pulse}
    Um Pulse mit Pulsdauern von wenigen Femtosekunden zu erzeugen werden Laser, die Licht mit einer besonders großen Bandbreite erzeugen, mit der Technik des Mode-Locking kombiniert. Die große Bandbreite
    $\Delta\nu$ ist dabei notwendig, da die zeitliche Dauer des Pulses $\Delta\tau$ durch die Heisenberg'sche Unschärferelation in Form des Zeit-Bandbreite-Produkts 
    
    \begin{equation}
      \Delta\tau \cdot \Delta\nu = \text{const},
    \end{equation}
    
    das eine von der Pulsform abhängige Konstante besitzt, nach unten limitiert ist.    


    \subsubsection{Mode-Locking}
      In dem Resonator eines Laser können bis zu 100000 longitudinale Moden entstehen, die im Dauerstrich Laserbetrieb ohne feste Phasenbeziehung schwingen. Beim Mode-Locking wird versucht zwischen den 
      einzelnen Moden eine feste Phasenbeziehung zu erreichen, sodass es zu Interferenz zwischen den stehenden Wellen kommen kann. Dies führt, wie in Abbildung \ref{fig:Modelocking} a) zu sehen, zu einer 
      periodischen Wiederholung der Intensität mit der Durchlaufzeit durch den Resonator $T_C$. Zur Optimierung der Interferenz wird versucht die Phase aller Moden anzugleichen, sodass, wie in 
      Abbildung \ref{fig:Modelocking} b) dargestellt, letztendlich nur noch ein einzelnes ultrakurzes Wellenpaket durch den Resonator wandert und ausgekoppelt werden kann. Das Wellenpaket setzt sich also
      aus den stehenden Wellen der verschiedenen Frequenzen und kann als Fouriertransformation des Frequenzspektrums gesehen werden. Dies erklärt, dass die Menge der Moden und dementsprechend die 
      Bandbreite $\Delta\nu$ die minimal mögliche Pulsdauer $\Delta\tau_{\text{ML}}$ nach

      \begin{equation*}
        \Delta\tau_{\text{ML}} = \frac{2\pi}{\Delta\nu}
      \end{equation*}

      bestimmt. 
      Umgesetzt werden kann das Modelocking entweder durch die aktive Modulation, bei der zum Beispiel ein Shutter aktiv oszilliert, um längere Pulse abzuschwächen, oder durch passive Modulationen. In dem hier 
      genutzen System wird mit der Additiven Puls Modenkopplung eine passive Modulationsart genutzt. Diese beruht darauf, dass zunächst elliptisch polarisiertes Licht im Dauerstrichbetrieb durch ein Medium 
      läuft, dessen Brechungsindex linear mit der Intensität des Lichts ansteigt. Wenn eine zufällige Fluktuation der Laserintensität die Form eines Pulses annimmt, führt der nicht lineare Brechungsindex 
      zu einer Rotation der Ellipse, da die lineare Komponente der Ellipse mit der stärkeren Amplitude stärker zeitlich verzögert und damit die Phasenverschiebung der beiden zueinander verändert wird. Diese
      Rotation ist für das Zentrum des Pules größer als für die weniger intensiven Anteile. Dieser Puls durchläuft nun ein $\lambda/4$-Plättchen, das den Puls linear polarisiert und anschließend eine  
      $\lambda/2$-Plättchen, das den Puls um einen Winkel dreht. Dieser Winkel ergibt für die Pulse maximaler Intensität einen Polarisationswinkel, sodass der Anteil maximaler Amplitude durch einen 
      Polarized Beam-Splitter Cube (PBSC) transmittiert wird. Für Pulsanteile geringerer Intensität ist der finale Winkel nicht optimal, sodass diese Anteile unterdrückt und effektiv abgeschnitten werden. Nach
      Transmission durch den PBSC wird der Puls erneut aus einer Kombination von einem $\lambda/2$- und $\lambda/4$-Plättchen elliptisch polarisiert. Die so erreichbaren Pulsdauern liegen im Bereich von 
      \SI{100}{\femto\second}, sofern der Strahlungsübergang des Lasers eine genügend große Linienbreite und somit viele Lasefähige Moden besitzt. 

      \FloatBarrier
      \begin{figure}[h]
        \centering
        \includegraphics[width = 0.44\textwidth]{pictures/ML_phase_const.png}
        \includegraphics[width = 0.44\textwidth]{pictures/ML_phase_gleich.png}
        \caption{a) Zeitlicher Verlauf der Intensität des Laser bei Überlagerung der Moden mit fester Phasenbeziehung. b)Zeitlicher Verlauf der Intensität, wenn die Moden alle die gleiche Phase besitzen. Entnommen aus \cite{tu_dortmund_versuchsanleitung_nodate}}
        \label{fig:Modelocking}
      \end{figure}
      \FloatBarrier
      
    \newpage
    \subsubsection{fs-Laser}

      Das in diesem Versuch genutzte und in Abbildung \ref{fig:Laser} dargestellte Lasersystem setzt sich aus einem Faserlaser zusammen, der mit einem Ringresonator in Kombination mit Verzögerungsplättchen 
      und eines PBSCs, einer undotierten Faser mit negativer Group-Velocity-Dispersion (GVD), einer dotierten Faser mit positiver GVD und weiteren Verzögerungsplättchen und Silizium Prismen zur Erzeugung von
      ultrakurzen Pulsen genutzt wird. 

      \FloatBarrier
      \begin{figure}[h]
        \centering
        \includegraphics[width = 0.9\textwidth]{pictures/Laser.png}
        \caption{Aufbau des Laser zur Erzeugung von ultrakurzen Pulsen. Entnommen aus \cite{tu_dortmund_versuchsanleitung_nodate}}
        \label{fig:Laser}
      \end{figure}
      \FloatBarrier

      Zur Erzeugung der primären Strahlung wird eine Quarzglas-Glasfaser genutzt, deren innerer Kern mit Er$^{3+}$-Ionen dotiert ist und als aktives Medium fungiert. Ein infraroter Laser mit einer Wellenlänge von 
      \SI{980}{\nano\metre} regt die Er$^{3+}$-Ionen in den höchsten Zustand eines Drei-Niveau-Systems an, dessen Übergang in den Grundzustand dipolverboten ist. Daher kann schnell eine Besetzungs erreicht 
      werden und durch Übergang in den zweithöchsten Zustand anschließend ein strahlender Übergang stattfinden. Dieser emitiert bei einer zentralen Wellenlänge von \SI{1550}{\nano\metre} und besitzt eine 
      große Bandbreite, da die zugehörigen Zustände durch die Elektron-Phonon-Wechselwirkung des dotierten inneren Kerns mit dem undotierten Quarzglas der äußeren Schicht stark verbreitert sind.
      
      Die so entstehende Strahlung befindet sich zunächst im Dauerstrichbetrieb und wird wie zuvor beschrieben über die Additiven Puls Modenkopplung in transformiert. Anschließend werden die 
  

      \FloatBarrier
      \begin{figure}[h]
        \centering
        \includegraphics[width = 0.9\textwidth]{pictures/Dispersion.png}
        \caption{Aufbau des Laser zur Erzeugung von ultrakurzen Pulsen. Entnommen aus \cite{tu_dortmund_versuchsanleitung_nodate}}
        \label{fig:Dispersion}
      \end{figure}
      \FloatBarrier
