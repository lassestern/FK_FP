\newpage
\section{Diskussion}    
    Die zur Kalibrierung der optischen Falle durchzuführenden Messungen lieferten größtenteils Ergebnisse, die starke Messfehler vermuten lassen.

    Die Untersuchung des Vesikeltransports innerhalb von Zwiebelzellen über die optische Pinzette, konnte mit geringem Aufwand beeindruckende Ergebnisse liefern. Die Charakterisierung der Größe und
    Geschwindigkeit der Vesikel stellte keine Problem dar. Besonders die dynamische Untersuchung des Vesikelverhaltens in Kombination mit Videoaufnahmen bot schnelle und tiefe Einblicke in die
    Charakteristik des Vesikeltransports. Alleinig die Messung der nötigen Fallensteifigkeit zum Stopp des Vesikeltransports liefert aufgrund der zuvor gescheiterten Kalibrierung der Fallensteifigkeit
    von der Laserleistung keine aussagekräftigen Ergebnisse.   

    Im Allgemeinen konnte das Konzept und Funktiosprinzip einer optischen Pinzette im Rahmen des Versuches gut erforscht werden. Während viele Messungen zur Kalibrierung der Falle ungenaue Ergebnisse liefern
    und nicht als erfolgreich zu bezeichnen sind, hat die phänomenologische Beobachtung des Vesikeltransports vollends funktioniert. 
