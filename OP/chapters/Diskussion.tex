\newpage
\section{Diskussion}
    Die zunächst durchgeführten Messungen zur Kalibrierung der Kamera und der Photodiode anhand bekannter Quarzkügelchen lieferten realistische Ergebnisse, die die Annahme einer korrekten Kalibrierung der 
    Detektionswege gerechtfertigen. Bei den von der genutzten Viersegmentdiode abhängigen Konversionsfaktoren zeigt sich ein für alle Laserleistungen kohärentes Ergebnis, sodass ein Mittelwert angenommen
    werden konnte. Die Messung des Diodensummensignals in Abhängigkeit der z-Position der optischen Falle bei eingefangenem Quarzkügelchen gibt den theoretischen monotonen Anstieg bis zum Streumaximum wieder, 
    der für Quarzkügelchen auftritt, die fest an der Oberfläche haften. Demnach wurde die Salzlösung korrekt präperiert.

    Die Kalibrierung der optischen Falle liefert für die unterschiedlichen Messaufbauten abweichende Ergebnisse. Bei der Messung ohne externe Kraft ist ein linearer Trend bei drei der fünf Messwerte
    erkennbar und die daran angepasste Gerade wird zur Kalibrierung der Fallensteifigkeit bei der Untersuchung des Vesikeltransports genutzt. Die aus den Fallensteifigkeiten berechneten Werte für die
    Boltzmannkonstante weichen maximal um 52\% vom Theoriewert von \SI{1.38}{\joule\per\kelvin} \cite{national_institute_of_standars_and_technology_nist_nodate} ab.\newline
    Die Messungen mit extern wirkender Kraft sollten dieselben oder zumindest vergleichbare Fallensteifigkeiten ergeben. Die berechneten Fallensteifigkeiten befinden sich wie erwartet in der selben 
    Größenordnung, weisen im Gegensatz aber keine Abhängigkeit von der Laserleistung auf. Dies liegt vermutlich an falsch gewählten Anregungsfrequenzen, die nicht 
    die Bedingung ans Verlieren des Quarzkügelchens aus der optischen Falle erfüllten. Auch wenn kein Zusammenhang der Fallensteifigkeiten zu erkennen ist, liefern diese ähnlich zur vorherigen Messung Werte 
    für die Boltzmannkonstante, die nicht mehr als 52\% von dem Literaturwert abweichen. Eine im Vergleich zur nicht vorhandenen externen Kraft neue Fehlerquelle, ist die periodische Bewegung der 
    Probenhalterung durch Piezomotoren. Die Piezoelemente sind zum Beispiel mit einer Hysterese behaftet und führen so nie genau die selbe Bewegung durch.\newline
    Die letzte Messung bei externer Kraft und eingesetztem Vortex Retarder liefert für die y-Richtung beinahe eine konstante Fallensteifigkeit, die wiederum Werte für die Boltzmannkonstante liefern, die alle 
    beinahe um 100\% vom Literaturwert abweichen. Die starke Abweichung zum Literaturwert der Boltzmannkonstante könnte am Vernachlässigen der Effekte liegen, die durch die Rotation des Quarzkügelchens 
    aufgrund des Drehimpulsübertrags bei eingesetztem Vortex-Retarder entstehen.\newline
    Alle Methoden zusammen lieferten eine gemittelte Boltzmann-Konstante von\newline \SI{9.899(4)e-24}{\joule\per\kelvin} , die um 99,99\% vom gegebenen Literaturwert abweicht. Zu dieser Abweichung können neben den 
    bereits genannten, für die Messreihen spezifischen Gründen auch die Detektion von Störsignalen durch nicht perfekt saubere Probenhalter oder durch die Anwesenheit von weiteren Quarzkügelchen im Bereich der 
    optischen Pinzette geführt haben. Die Asymetrie der Werte zwischen x- und y-Richtung könnte in einer Asymetrie der Apparatur, in Form der Piezomotoren zur Erzeugung der externen Kraft oder einer nicht 
    perfekt ausgerichteten Vier-Segment-Photodiode, begründet sein. Aber auch eine nicht perfekt horizontale Ausrichtung der Probe zum Boden könnte zu einer Asymetrie der Kräfte und somit Messergebnisse 
    geführt haben.  Zusätzlich werden die Quarzkügelchen in der optischen Falle als unabhängig angesehen. Innerhalb der Falle kommt es jedoch unter anderem aufgrund der thermischen Erhitzung zu 
    Konvektionsströmen \cite{hosokawa_convection_2020}, die die Kügelchen ebenfalls beeinflussen und nicht berücksichtigt worden sind.  

    Die Untersuchung des Vesikeltransports innerhalb von Zwiebelzellen über die optische Pinzette konnte mit geringem Aufwand beeindruckende Ergebnisse liefern. Die Charakterisierung der Größe und
    Geschwindigkeit der Vesikel stellte keine Problem dar. Besonders die dynamische Untersuchung des Vesikelverhaltens in Kombination mit Videoaufnahmen bot schnelle und tiefe Einblicke in die
    Charakteristik des Vesikeltransports.

    Im Allgemeinen konnte das Konzept und Funktionsprinzip einer optischen Pinzette im Rahmen des Versuches gut erforscht werden. Auch wenn mit fortschreitender Komplexität in Form der externen Krafteinwirkung
    und des eingesetzten Vortex-Retarders vermehrt Abweichungen aufgrund neuer Fehlerquellen, wie der Bewegung der Piezomotoren, und Vernachlässigung von Effekten wie des Drehimpulsübertrags bei
    eingesetztem Vortex-Retarders, auftraten, konnten die Messungen qualitativ erfolgreich durchgeführt werden. Die komplett phänomenologische Beobachtung des Vesikeltransports hat demnach auch 
    sehr gut funktioniert. 
