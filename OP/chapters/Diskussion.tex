\newpage
\section{Diskussion}
    Die zunächst durchgeführten Messungen zur Kalibrierung der Kamera und der Photodiode lieferten vertretbare Ergebnisse, die die Auswertug der Daten der beiden Instrumente ermöglichen. Bei den 
    Konversionsfaktoren wird aufgrund keines klaren Zusammenhangs zu der Laserleistung ein Mittelwert angenommen.
    Die Kalibrierung der optischen Falle liefert für die unterschiedlichen Messaufbauten abweichende Ergebnisse. Bei der Messung ohne externe Kraft ist der lineare Trend bei drei der fünf Messwerte
    erkennbar und die daran angepasste Gerade wird zur Kalibrierung der Fallensteifigkeit bei der Untersuchung des Vesikeltransports genutzt. Die aus den Fallensteifigkeiten berechneten Werte für die
    Boltzmannkonstante weichen maximal um 52\% vom Theoriwert von \SI{1.38}{\joule\per\kelvin} (\cite{national_institute_of_standars_and_technology_nist_nodate}) ab. Die Messungen mit extern wirkender Kraft
    ergeben Fallensteifigkeiten, die keine Abhängigkeit von der Laserleistung aufweisen. Dies liegt vermutlich an falsch gewählten Anregungsfrequenzen, die nicht die Bedingung ans Verlieren des Quarzkügelchens
    aus der optischen Falle erfüllten. Auch wenn kein linearer Zusammenhang zu erkennen ist, liefern die Fallensteifigkeiten ähnlich zur vorherigen Messung Werte für die Boltzmannkonstante, die nicht mehr als
    52\% von dem Literaturwert abweichen. Die letzte Messung bei externer Kraft und eingesetztem Vortex Retarder liefert für die y-Richtung beinahe eine konstante Fallensteifigkeit, die wiederum Werte für
    die Boltzmannkonstante liefern, die alle beinahe um 100\% vom Literaturwert abweichen. Die x-Komponente gibt den erwarteten Anstieg mit der Laserleistung wieder und liefert auch passendere, nicht mehr als 
    61\% Prozent vom Literaturwert abweichende, Werte der Boltzmannkonstante.

    Die Untersuchung des Vesikeltransports innerhalb von Zwiebelzellen über die optische Pinzette konnte mit geringem Aufwand beeindruckende Ergebnisse liefern. Die Charakterisierung der Größe und
    Geschwindigkeit der Vesikel stellte keine Problem dar. Besonders die dynamische Untersuchung des Vesikelverhaltens in Kombination mit Videoaufnahmen bot schnelle und tiefe Einblicke in die
    Charakteristik des Vesikeltransports.

    Im Allgemeinen konnte das Konzept und Funktiosprinzip einer optischen Pinzette im Rahmen des Versuches gut erforscht werden. Während viele Messungen zur Kalibrierung der Falle ungenaue Ergebnisse liefern
    und nicht als erfolgreich zu bezeichnen sind, hat die phänomenologische Beobachtung des Vesikeltransports vollends funktioniert. 
